\section{Partial Optimality for Cubic Clique Partition Problem}
\frame{\tableofcontents[currentsection]}


\begin{frame}
    \frametitle{Partial Optimality for Cubic Clique Partition Problem}
    % mention triple, pairs and constant!!!
    % fix labels if something holds
    % improving map
    % reformulate the problem
    % 2 correct join conditions cannot be combined in the optimal solution
    % so when applied simultaneously they result into a suboptimal solution. 

    \onslide<1->{
        Construct \textbf{Improving Maps} for the clustering $\labels$
        in the cubic clique partition problem} \onslide<2->{
    $\to$
    \textbf{Partial Optimality Conditions:}
    \begin{enumerate}
        \onslide<3->{
        \item Subproblem-CUT-condition (cut subset from its complement)
        }
        \onslide<4->{
        \item CUT-conditions (cut pairs and triples)
        }
        \onslide<5->{
        \item JOIN-conditions (join subsets, pairs and triples)\\
        }
    \end{enumerate}
    }
    \vspace{5px}
    \onslide<6->{
    CUT-conditions can be applied simultaneously\\
    JOIN-conditions cannot be applied simultaneously!\\
    }
    \vspace{5px}
    \onslide<7->{
    \textbf{Partial Optimality Algorithm:\\}
    \hspace{10px}\textbf{Input:} clustering $\labels$ without fixed labels
    \begin{algorithmic}
        \While{condition applied}
            \State apply subproblem-CUT-condition exhaustively
            \State apply one of JOIN-conditions (in effective order)
        \EndWhile
        \State apply CUT-conditions exhaustively
    \end{algorithmic}
    \hspace{10px}\textbf{Output:} partially optimal clustering $\labels$ with some fixed labels
    }
\end{frame}


\begin{frame}
    \frametitle{Program Structure}
    Class Diagram
    Algorithm implementation in ClusteringProblem
    Features: ClusteringProblem is generally defined for all types of Cubic Clique Partition Problem (not necessarily points),
        cost function + sparse costs!, label computation, cut triples,
        logs joins and cuts! (add screenshots)
\end{frame}


\begin{frame}
    \frametitle{Subproblem-CUT}
    A couple of words about the split and the implementation
    (with picture of splitting)
\end{frame}


\begin{frame}
    \frametitle{JOIN-conditions}
    Special attention to 3.11 (+ my adjustment)
    Mention reduction to Min-Cut problem and the complexity!!!
\end{frame}


\begin{frame}
    \frametitle{JOIN-conditions}
    Overview of the other join-conditions (with pictures)
\end{frame}

\begin{frame}
    \frametitle{CUT-conditions}
    Overview of the cut-conditions (with pictures)
\end{frame}

\begin{frame}
    \frametitle{Example}
    Pyramid example for my algorithm
\end{frame}