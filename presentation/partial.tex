\section{Partial Optimality for Cubic Clique Partition Problem}
\frame{\tableofcontents[currentsection]}


\begin{frame}
    \frametitle{Partial Optimality for Cubic Clique Partition Problem}
    % mention triple, pairs and constant!!!
    % fix labels if something holds
    % improving map
    % reformulate the problem
    % 2 correct join conditions cannot be combined in the optimal solution
    % so when applied simultaneously they result into a suboptimal solution. 
    \onslide<1->{
        Extended cost function
        $\cost \colon \binom{\fSet}{3} \cup \binom{\fSet}{2} \cup {\emptyset} \to \real$\\
    }
    \vspace{5px}
    \onslide<2->{
        Instance of the extended cubic clique partition problem:
        \[
            \min\limits_{\labels \colon \binom{\fSet}{2} \to \bSet}
            \sum\limits_{\{a,b,c\} \in \binom{\fSet}{3}}
            \cost_{\{a,b,c\}}\labels_{\{a,b\}}\labels_{\{b,c\}}\labels_{\{a,c\}} 
            + \cost_{\{a,b\}}\labels_{\{a,b\}}
            + \cost_{\emptyset}
        \]\\
    }
    \vspace{5px}
    \onslide<3->{
        Construct \textbf{Improving Maps} for the clustering $\labels$\\
    }
    \vspace{5px}
    \onslide<4->{ 
    $\to$ \textbf{Partial Optimality Conditions:}
    \vspace{5px}
    \begin{enumerate}
        \onslide<5->{
        \item Subproblem-CUT-condition (cut subset from its complement)
        }
        \onslide<6->{
        \item CUT-conditions (cut pairs and triples)
        }
        \onslide<7->{
        \item JOIN-conditions (join subsets, pairs and triples)\\
        }
    \end{enumerate}
    }
    \vspace{5px}
    \onslide<8->{
    CUT-conditions can be applied simultaneously\\
    JOIN-conditions cannot be applied simultaneously!\\
    }
    \vspace{5px}
    \onslide<9->{
    Apply partial optimality connditions $\to$ solve subproblems\\    
    }
\end{frame}

\begin{frame}
    \frametitle{Partial Optimality Algorithm}
    \onslide<1->{
    \textbf{Partial Optimality Algorithm:\\}
    \hspace{10px}\textbf{Input:} clustering $\labels$ without fixed labels
    \begin{algorithmic}
        \While{condition applied}
            \State apply subproblem-CUT-condition exhaustively
            \State apply one of JOIN-conditions (in effective order)
        \EndWhile
        \State apply CUT-conditions exhaustively
    \end{algorithmic}
    \hspace{10px}\textbf{Output:} partially optimal clustering $\labels$ with some fixed labels
    }
    \vspace{10px}
    \onslide<2->{
    \textbf{Reduction to subproblems:}
    \begin{enumerate}
        \item Subproblem-CUT-condition:
        fix CUT labels for element pairs from different sample subsets; 
        solve each subset as an independent problem and accumulate the results in $\cost_{\emptyset}$;
        \onslide<3->{
        \item JOIN-Conditions: 
        fix JOIN labels for elements of the sample subset; 
        add the join-cost to $\cost_{\emptyset}$;
        solve the problem where the subset is considered as one sample;
        }
    \end{enumerate}
    }
\end{frame}

\begin{frame}
    \frametitle{Subproblem-CUT and Subset-JOIN}
    % A couple of words about the split and the implementation
    % apply exhaustively
    % intuition 3.1 and 3.11!!!
    % 3.11 adjustment
    \onslide<1->{
    \textbf{Subproblem-CUT:} cut sample subsets $\subR_1, \subR_2, \dots, \subR_k$
    that are only connected via non-negative costs (applied if $k > 1$)\\
    }
    \vspace{5px}
    \onslide<2->{
    \textbf{Subset-JOIN:} join sample subset $\subR$ with only non-positive costs
    if its worst bipartition joining cost is less than or equal to
    the reward of joining $\subR$ with $\overline{\subR}$ (applied if $|\subR| > 1$)\\
    (The worst bipartition joining cost $\approx$ min-cut)\\
    }
    \vspace{15px}
    \onslide<3->{
        \begin{tikzpicture}[node distance=30px]
            % nodes
            \node[state] (a) {a};
            \node[state] (b) [below left of=a] {b};
            \node[state] (c) [below right of=b] {c};
            \node[state] (d) [above right of=c] {d};
            \node[state] (e) [below right of=d] {e};
            \node[state] (f) [above right of=e] {f};
            \node[state] (h) [below of=e] {h};
            \node[state] (g) [above of=d] {g};
            \node[state] (i) [above of=f] {i};
            % edges
            \path[-] (a) edge node[below right=3px, below] {\footnotesize -1} (b);
            \path[-] (a) edge node {} (c);
            \path[-] (b) edge node {} (c);
            \path[-] (a) edge node[below left=3px, below] {\footnotesize -15} (d);
            \path[-] (c) edge node {} (d);
            \path[-] (d) edge node {} (g);
            \path[-] (d) edge node {} (f);
            \path[-] (g) edge node {} (i);
            \path[-] (i) edge node[above left] {\footnotesize -10} (f);
            \path[-] (g) edge node[below left] {\footnotesize -2} (f);
            \path[-] (d) edge node {} (e);
            \path[-] (e) edge node {} (f);
            \path[-] (e) edge node {} (h);
            \path[-] (d) edge node[right=-3px] {\tiny 50} (h);
            \path[-] (h) edge node[left=-4px] {\tiny -50} (f);
        \end{tikzpicture}
    }
    \hspace{5px}
    \onslide<4->{
        \begin{tikzpicture}[node distance=30px]
            % nodes
            \node[state] (acd) {acd};
            \node[state] (b) [left of=acd] {b};
            \node[state] (e) [below right of=acd] {e};
            \node[state] (f) [above right of=e] {f};
            \node[state] (h) [below of=e] {h};
            \node[state] (g) [above of=acd] {g};
            \node[state] (i) [above of=f] {i};
            % edges
            \path[-] (acd) edge node[above] {\footnotesize -1} (b);
            \path[-] (acd) edge node {} (g);
            \path[-] (acd) edge node {} (f);
            \path[-] (g) edge node {} (i);
            \path[-] (i) edge node[above left] {\footnotesize -10} (f);
            \path[-] (g) edge node[below left] {\footnotesize -2} (f);
            \path[-] (acd) edge node {} (e);
            \path[-] (e) edge node {} (f);
            \path[-] (e) edge node {} (h);
            \path[-] (acd) edge node[right=-3px] {\tiny 50} (h);
            \path[-] (h) edge node[left=-4px] {\tiny -50} (f);
        \end{tikzpicture}
    }
    \onslide<5->{
        \begin{tikzpicture}[node distance=30px]
            % nodes
            \node[state] (acd) {acd};
            \node[state] (b) [above of=acd] {b};
            \node[state] (efghi) [below of=acd] {efghi};
            % edges
            \path[-] (acd) edge node[right] {\footnotesize -1} (b);
            \path[-] (acd) edge node[right] {\footnotesize 48} (efghi);
        \end{tikzpicture}
    }
    \onslide<6->{
        \begin{tikzpicture}[node distance=30px]
            % nodes
            \node[state] (acd) {acd};
            \node[state] (b) [above of=acd] {b};
            \node[state] (efghi) [below of=acd] {efghi};
            % edges
            \path[-] (acd) edge node[right] {\footnotesize -1} (b);
        \end{tikzpicture}
    }
    \onslide<7->{
        \begin{tikzpicture}[node distance=30px]
            % nodes
            \node[state] (abcd) {abcd};
            \node[state] (efghi) [below of=acd] {efghi};
        \end{tikzpicture}
    }
    \\
    \hspace{20px}
    \onslide<3->{$\cost_{\emptyset}=0$}
    \hspace{60px}
    \onslide<4->{$\cost_{\emptyset}=-15$}
    \hspace{25px}
    \onslide<5->{$\cost_{\emptyset}=-75$}
    \hspace{15px}
    \onslide<7->{$\cost_{\emptyset}=-76$}
\end{frame}


\begin{frame}
    \frametitle{Other JOIN-conditions and CUT-conditions}
    \onslide<1->{
        \textbf{Pair-JOIN-1:} join samples $i,j$ if
        their overall joining reward $\geq$
        the sum of rewards and penalties for joining
        some subset $\subR$ with $i \in \subR$ 
        and $\overline{\subR}$ with $j \in \overline{\subR}$
        (rhs $\approx$ min-cut)\\
    }
    \vspace{10px}
    \onslide<2->{
        \textbf{Pair-JOIN-2:} join samples $i,k$ if 
        there exists a sample triple $ijk$ that fulfills 
        3 conditions
        (2 conditions $\approx$ min-cut, 1 explicit condition)\\
    }
    \vspace{10px}
    \onslide<3->{
        \textbf{Pair-JOIN-3:} join samples $i,j$ if
        $\cost_{\{i,j\}}\leq$ 
        the sum of reward costs for joining pairs and triples 
        containing $i$ or $j$\\
    }
    \vspace{10px}
    \onslide<4->{
        \textbf{Pair-JOIN-4:} join samples $i,k$
        if there exists a sample triple $ijk$ such that 
        7 explicit conditions hold\\
    }
    \vspace{10px}
    \onslide<5->{
        \textbf{Triple-JOIN:} join samples $i,j,k$ 
        if a condition holds\\ 
        (similar to Pair-JOIN-1)
        (rhs $\approx$ min-cut)\\
    }
    
\end{frame}


\begin{frame}
    \frametitle{Pyramid Example and CUT-conditions}
    \vspace{-5px}
    $\cost_{\{b,c,d\}}=10$\\
    $\cost_{\{a,b,c\}}=\cost_{\{a,b,d\}}=\cost_{\{a,c,d\}}=-50$\\
    \vspace{5px}
    \onslide<1->{    
        \begin{tikzpicture}[node distance=30px]
            % nodes
            \node[state] (a) {a};
            \node[state] (b) [below left of=a, below=5px] {b};
            \node[state] (c) [below right of=a, right=10px] {c};
            \node[state] (d) [below right of=b] {d};
            % edges
            \path[-] (a) edge node {} (b);
            \path[-] (a) edge node {} (c);
            \path[-] (a) edge node {} (d);
            \path[-, dashed] (b) edge node {} (c);
            \path[-] (c) edge node {} (d);
            \path[-] (d) edge node {} (b);
        \end{tikzpicture}
    }
    \onslide<2->{ 
        \hspace{20px}   
        \begin{tikzpicture}[node distance=30px]
            % nodes
            \node[state] (ac) {ac};
            \node[state] (b) [below left of=a] {b};
            \node[state] (d) [below right of=a] {d};
            % edges
            \path[-] (ac) edge node[above left=-3px] {\footnotesize-50} (b);
            \path[-] (ac) edge node[above right=-3px] {\footnotesize-50} (d);
            \path[-] (d) edge node[above] {\footnotesize-40} (b);
        \end{tikzpicture}
    }
    \hspace{10px}
    \onslide<3->{ 
        \hspace{30px} 
        \begin{tikzpicture}[node distance=30px]
            % nodes
            \node[state] (abcd) {abcd};
        \end{tikzpicture}   
    }\\
    \hspace{40px}
    \onslide<2->{Pair-JOIN-2}
    \hspace{45px}
    \onslide<3->{
        Subset-JOIN
        \hspace{10px}
        $\cost_{\emptyset}=-140$
    }
    
    \vspace{10px}
    \onslide<4->{
        \textbf{Pair-CUT:} cut samples $i,j$ if
        the direct joing penalty $\geq$
        the sum of rewards for joining
        some subset $\subR$ with $i \in \subR$ 
        and $\overline{\subR}$ with $j \in \overline{\subR}$
        (rhs $\approx$ min-cut)\\
    }
    \vspace{5px}
    \onslide<5->{
        \textbf{Triple-CUT:}join samples $i,j,k$ 
        if a condition holds\\ 
        (similar to Pair-CUT)
        (rhs $\approx$ min-cut)\\
    }
    \vspace{10px}
    \onslide<6->{
        Samples in pyramid with $\cost_{\{b,c,d\}}=100$ are unjoinable!\\
        Triple-CUT is applied to the triple $bcd$
    }
\end{frame}


\begin{frame}
    \frametitle{Program Structure}
    % tested on the examples i have shown you before
    TODO: class Diagram
    TODO: features
    Algorithm implementation in ClusteringProblem
    Features: ClusteringProblem is generally defined for all types of Cubic Clique Partition Problem (not necessarily points),
        cost function + sparse costs!, label computation, cut triples,
        logs joins and cuts! (add screenshots)
\end{frame}





